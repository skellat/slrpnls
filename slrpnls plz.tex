% arara: lualatex: { shell: yes }
% arara: lualatex: { shell: yes }
% arara: lualatex: { shell: yes }
% arara: lualatex: { shell: yes }
% arara: clean: { extensions: [ aux, log, out, toc, ent ] }

% !TeX program = lualatex
% !TeX encoding = UTF-8

\documentclass[
notumble,
%%nofoldmark,
%%dvipdfm,
%%portrait,
%%titlepage,
%%nocombine,
%%a3paper,
%%debug,
%%nospecialtricks,
%%draft,
letterpaper
]{leaflet}

%\usepackage{tgpagella}
%\usepackage{venturis2}
%\usepackage{forum}
%\usepackage{stix}
\usepackage{tgtermes}
\usepackage[T1]{fontenc}
\let\oldstylenums\oldstyle
\usepackage{hyperref}
\usepackage{xcolor}
\usepackage{etoolbox}
\usepackage{keyval}
\usepackage{ifthen}
\usepackage{csquotes}
\usepackage{hyperxmp}
\usepackage[type={CC},modifier={by-sa},version={4.0},]{doclicense}
\usepackage{uri}
\usepackage{latexsym}
\usepackage{graphicx}
\usepackage{listings}
\usepackage{evenpage}
\usepackage{microtype}
\usepackage{graphicx}
\usepackage{relsize}
\usepackage{markdown}
\usepackage{termsim}
\usepackage{titlepic}
\usepackage{rpgicons}
\usepackage{customdice}
\usepackage{utfsym}
\usepackage{fontawesome5}
\usepackage{textcomp}
\usepackage{url}
\usepackage[svgnames]{xcolor}
%\usepackage[paperheight=11in,paperwidth=8.5in,top=1in,bottom=1in,right=1in,left=1in]{geometry}
\DeclareGraphicsExtensions{.pdf,.png,.jpg}

\hypersetup{
    pdftitle={slrpnls plz},    % title
    pdfauthor={Stephen Michael Kellat},     % author
    pdfsubject={TTRPG},   % subject of the document
    colorlinks=true,       % false: boxed links; true: colored links
   pdfdisplaydoctitle=true, %Show document title instead of file name in title bar
   pdfnewwindow=true, %open links in a new window
    pdfpagemode=UseOutlines,
    bookmarksopen=true,
    pdfkeywords={Games},
    pdfpagelayout=OneColumn,
    allcolors={black},
    pdfstartview=FitH,
}


\title{slrpnls plz}
\author{Stephen Michael Kellat}
\date{}

\begin{document}

\maketitle

\maketitle

\begin{center}
\begin{Huge}\usym{1F4F7} \\
\usym{1F4F9} \\
\usym{1F4F8} \\
\usym{1F3A5} \\
\end{Huge}
\end{center}

\section*{Introduction}

Do you ever watch low-budget and no-budget films?  In the era of streaming television they're everywhere.  It is pretty easy to make and distribute your own film today.  Everybody can be their own King or Queen of Cult if they truly want to try.

So, let's try!  This can be played solo or with a GM acting as "The Financier" with a group of players.
\cleardoublepage
\section*{Determining Your Budget}

Your Financier needs to make some rolls to start off.  A budget cap has to be set.  You will need:

\begin{center}
    \begin{tabular}{ |c|c| }
        \hline
        Die & Description \\
        \hline
        \die[large]{twentyside}{20} & Twenty sided die \\
        \hline
        \die[large]{twelveside}{12} & Twelve sided die \\
        \hline
        \die[large]{tenside}{10} & Ten sided die \\
        \hline
        \die[large]{eightside}{8} & Eight sided die \\
        \hline
        \die[large]{fourside}{4} & Four sided die \\
        \hline
    \end{tabular}   
\end{center}

The Financier rolls each die once and writes the numbers down.  He then multiplies the numbers together.  That total is then multipled by one. hundred.  That gives you the budget topline you just cannot exceed with your proposed film.  If it seems low, it likely is.  The dice are helping you craft a story as efficiently as possible.

Once you have your budget topline, you can then move on to setting a few more parameters for your story.

\begin{center}
\begin{Huge}\usym{1F4B0} \\
\faMoneyBill* \\
\faMoneyBillWave \\
\faMoneyCheck* \\
\end{Huge}
\end{center}
\cleardoublepage
\section*{The Tables of Wonders}

Cheesy low-budget/no-budget films generally stick to certain genres and tropes.  Rolling on the tables below lets you mix and match those genres and tropes to come up with your own unique twist for a new film.  You will need a \die[large]{sixside six}~six sided die to roll three separate times.

\begin{center}
\begin{tabular}{ |c|c| } 
 \hline
 Roll & Genre \\ 
 \hline
 \die[large]{sixside}{1} & Sword and Sandal \\
 \hline
 \die[large]{sixside}{2} & Comedy \\
 \hline
 \die[large]{sixside}{3} & Thriller \\
 \hline
 \die[large]{sixside}{4} & Martial Arts \\
 \hline
 \die[large]{sixside}{5} & Whodunnit \\
 \hline
 \die[large]{sixside}{6} & Exploitation \\
 \hline
\end{tabular}
\end{center}

\begin{center}
\begin{tabular}{ |c|c| } 
 \hline
 Roll & Twist Location \\ 
 \hline
 \die[large]{sixside}{1} & \ldots in Space \\
 \hline
 \die[large]{sixside}{2} & \ldots in  Europe \\
 \hline
 \die[large]{sixside}{3} & \ldots in the Southern USA \\
 \hline
 \die[large]{sixside}{4} & \ldots in the South Pacific \\
 \hline
 \die[large]{sixside}{5} & \ldots in a world of science fiction \\
 \hline
 \die[large]{sixside}{6} & \ldots in a cyberpunk dystopia \\
 \hline
\end{tabular}
\end{center}

\begin{center}
\begin{tabular}{ |c|c| } 
 \hline
 Roll & Era \\ 
 \hline
 \die[large]{sixside}{1} & 1970s \\
 \hline
 \die[large]{sixside}{2} & 1990s \\
 \hline
 \die[large]{sixside}{3} & The Present \\
 \hline
 \die[large]{sixside}{4} & Antebellum America \\
 \hline
 \die[large]{sixside}{5} & A Cyberpunk Future \\
 \hline
 \die[large]{sixside}{6} & 20 Minutes Into The Future \\
 \hline
\end{tabular}
\end{center}
\cleardoublepage

\section*{What Next?}

After you have
\clearpage
\clearpage
\doclicenseThis

\end{document}




