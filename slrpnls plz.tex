% arara: lualatex: { shell: yes }
% arara: lualatex: { shell: yes }
% arara: lualatex: { shell: yes }
% arara: lualatex: { shell: yes }
% arara: clean: { extensions: [ aux, log, out, toc, ent ] }

% !TeX program = lualatex
% !TeX encoding = UTF-8

\documentclass[
notumble,
%%nofoldmark,
%%dvipdfm,
%%portrait,
%%titlepage,
%%nocombine,
%%a3paper,
%%debug,
%%nospecialtricks,
%%draft,
%%letterpaper
a5paper
]{leaflet}

%\usepackage{tgpagella}
\usepackage{venturis2}
%\usepackage{forum}
%\usepackage{stix}
%\usepackage{tgtermes}
\usepackage[T1]{fontenc}
\let\oldstylenums\oldstyle
\usepackage{hyperref}
\usepackage{xcolor}
\usepackage{etoolbox}
\usepackage{keyval}
\usepackage{ifthen}
\usepackage{csquotes}
\usepackage{hyperxmp}
\usepackage[type={CC},modifier={by-sa},version={4.0},]{doclicense}
\usepackage{uri}
\usepackage{latexsym}
\usepackage{graphicx}
\usepackage{listings}
\usepackage{evenpage}
\usepackage{microtype}
\usepackage{graphicx}
\usepackage{relsize}
\usepackage{markdown}
\usepackage{termsim}
\usepackage{titlepic}
\usepackage{rpgicons}
\usepackage{customdice}
\usepackage{utfsym}
\usepackage{fontawesome5}
\usepackage{textcomp}
\usepackage{url}
\usepackage{hologo}
\usepackage[svgnames]{xcolor}
%\usepackage[paperheight=11in,paperwidth=8.5in,top=1in,bottom=1in,right=1in,left=1in]{geometry}
\DeclareGraphicsExtensions{.pdf,.png,.jpg}

\hypersetup{
    pdftitle={slrpnls plz},    % title
    pdfauthor={Stephen Michael Kellat},     % author
    pdfsubject={TTRPG},   % subject of the document
    colorlinks=true,       % false: boxed links; true: colored links
   pdfdisplaydoctitle=true, %Show document title instead of file name in title bar
   pdfnewwindow=true, %open links in a new window
    pdfpagemode=UseOutlines,
    bookmarksopen=true,
    pdfkeywords={Games},
    pdfpagelayout=OneColumn,
    allcolors={black},
    pdfstartview=FitH,
}


\title{slrpnls plz}
\author{Stephen Michael Kellat}
\date{}

\begin{document}

\maketitle

\maketitle

\begin{center}
\begin{Large}\usym{1F4F7} \usym{1F4F9} \usym{1F4F8} \usym{1F3A5} \end{Large}
\end{center}

\section*{Introduction}

Do you ever watch low-budget and no-budget films?  In the era of streaming television they're everywhere.  It is pretty easy to make and distribute your own film today.  Everybody can be their own King or Queen of Cult if they truly want to try.

So, let's try!  This is played by a GM acting as "The Financier" with a group of up to four players.
\cleardoublepage
\section*{Determining Your Budget}

Your Financier needs to make some rolls to start off.  A budget cap has to be set.  You will need a normal set of polyhedral dice used for roleplaying games as well as additional six-sided dice.  The Financier will be rolling the D20, the D12, the D10, the D8, and the D4 at the start.

The Financier rolls each die once and writes the numbers down.  He then multiplies the numbers together.  That total is then multipled by one hundred.  That gives you the budget topline you just cannot exceed with your proposed film.  If it seems low, it likely is.  The dice are helping you craft a story as efficiently as possible.

Once you have your budget topline, you can then move on to setting a few more parameters for your story.

\begin{center}
\begin{Large}\usym{1F4B0} \faMoneyBill* \faMoneyBillWave \faMoneyCheck* \end{Large}
\end{center}
\cleardoublepage
\section*{The Tables of Wonders}

Cheesy low-budget/no-budget films generally stick to certain genres and tropes.  Rolling on the tables below lets you mix and match to come up with your own unique twist for a new film.  You will need to roll a D6 five separate times.

\begin{center}
\begin{tabular}{ |c|c| } 
 \hline
 Roll & Genre \\ 
 \hline
 1 & Sword and Sandal \\
 \hline
 2 & Comedy \\
 \hline
 3 & Thriller \\
 \hline
 4 & Martial Arts \\
 \hline
 5 & Whodunnit \\
 \hline
 6 & Exploitation \\
 \hline
\end{tabular}
\end{center}

\begin{center}
\begin{tabular}{ |c|c| } 
 \hline
 Roll & Twist Location \\ 
 \hline
 1 & \ldots in Space \\
 \hline
 2 & \ldots in  Europe \\
 \hline
 3 & \ldots in the Southern USA \\
 \hline
 4 & \ldots in the South Pacific \\
 \hline
 5 & \ldots in a world of science fiction \\
 \hline
 6 & \ldots in a cyberpunk dystopia \\
 \hline
\end{tabular}
\end{center}

\begin{center}
\begin{tabular}{ |c|c| } 
 \hline
 Roll & Era \\ 
 \hline
 1 & 1970s \\
 \hline
 2 & 1990s \\
 \hline
 3 & The Present \\
 \hline
 4 & Antebellum America \\
 \hline
 5 & A Cyberpunk Future \\
 \hline
 6 & 20 Minutes Into The Future \\
 \hline
\end{tabular}
\end{center}

\begin{center}
\begin{tabular}{ |c|c| } 
 \hline
 Roll & Characters \\ 
 \hline
 1 & Beatniks \\
 \hline
 2 & Confused Lovers \\
 \hline
 3 & Unprepared Students \\
 \hline
 4 & Taster's Choice of Dirtbags \\
 \hline
 5 & Aliens \\
 \hline
 6 & Robots \\
 \hline
\end{tabular}
\end{center}

\begin{center}
\begin{tabular}{ |c|c| } 
 \hline
 Roll & Plot Driver \\ 
 \hline
 1 & Stopping Exploitation \\
 \hline
 2 & Carrying Out A Spy Mission \\
 \hline
 3 & Being A Private Detective \\
 \hline
 4 & Searching For Monsters \\
 \hline
 5 & Going To Outer Space \\
 \hline
 6 & The Search For The Holy Symbol \\
 \hline
\end{tabular}
\end{center}

\cleardoublepage

\section*{What Next?}

After you have a budget cap and rolled your premise, you need to start coming up with your story idea.  No story idea is too half-baked or off the wall.  Remember, your budget cap is way too low.  Whatever you want to do is going to have cut corners \emph{somewhere}.  Leverage that.

When your group is done creating their story ideas they need to come up with pitches.  The financier will hear the players out as to their proposed films.  Optimally one idea should be chosen but that doesn't have to be the case.  Not all ideas are good ideas.  How you choose is a group decision to have been made at the start.

At the end, celebrate your success.  Imagine how you would make such films.  If you're bold enough, take those sorts of ideas and processes to the next Film Race and see what you can do.  After all, we need more stories being told in our world.


\clearpage
\clearpage
\doclicenseThis

\begin{center}
Proudly produced using \hologo{LuaLaTeX} by Erie Looking Productions in Ashtabula, Ohio.  Support us via Ko-Fi at \url{https://ko-fi.com/smkellat}.
\end{center}
\end{document}




